\label{sec:EBD-installation}

\section{Installing the Elastic Body Dynamics Tool}

Download the installation package for your operation system from (a single line)

\begin{verbatim}
https://www.designsafe-ci.org/data/browser/public/designsafe.storage.community/
            /SimCenter/Software/elasticBodyDynamicsTool
\end{verbatim}

SimCenter is providing packages for Windows~8/10 (64 bit version only) and MacOS. The installer will place the executable on your system.  On Windows systems, a shortcut will be added to your start menu. On MacOS, the application is placed in your Applications folder.

\bigskip

For Linux systems, you will need to clone the source from 

\begin{verbatim}
https://github.com/NHERI-SimCenter/elasticBodyDynamicsTool
\end{verbatim}

and compile it yourself performing the following steps:

\begin{quote}
\begin{verbatim}
$ git clone https://github.com/NHERI-SimCenter/elasticBodyDynamicsTool
$ git clone https://github.com/NHERI-SimCenter/SimCenterCommon
$ cd elasticBodyDynamicsTool
$ qmake elasticBodyDynamicsTool.pros
$ make
$ sudo make install
\end{verbatim}
\end{quote}

\section{Compiling the Source Code in OpenFOAM}

Download the source code of the elastic body dynamics library from

\begin{verbatim}
https://github.com/NHERI-SimCenter/elasticBodyDynamicsTool/elasticBodyDynamics
\end{verbatim}

\noindent Note that the code is provided for OpenFOAM version 7. So please install the correct OpenFOAM version following the instructions from

\begin{verbatim}
https://openfoam.org/release/7/
\end{verbatim}

\noindent Create a project directory named as \textcolor{blue}{run} within the \textcolor{blue}{\$HOME/OpenFOAM} directory named \textless{USER}\textgreater{-7} (e.g. SimCenter-7 for user SimCenter and OpenFOAM version 7) by typing the following script in a terminal prompt:

\begin{quote}
\begin{verbatim}
$ mkdir -p $FOAM_RUN
\end{verbatim}
\end{quote}

\noindent Copy or move the \textcolor{blue}{elasticBodyDynamics} directory which has been downloaded earlier and all the files in it to the \textcolor{blue}{\$HOME/OpenFOAM} directory. Go the relocated \textcolor{blue}{elasticBodyDynamics/src} directory by typing:

\begin{quote}
\begin{verbatim}
$ cd $FOAM_RUN/elasticBodyDynamics/src
\end{verbatim}
\end{quote}

\noindent Compile the files in the \textcolor{blue}{elasticBodyDynamics/src} directory by typing the following in the terminal prompt:

\begin{quote}
\begin{verbatim}
$ wmake
\end{verbatim}
\end{quote}

\noindent After the compilation is successfully complete, a library file named as \textcolor{blue}{libelasticBodyDynamics.so} will be generated in the directory \textcolor{blue}{\$FOAM\_USER\_LIBBIN}.

\noindent Go the relocated \textcolor{blue}{elasticBodyDynamics/applications/pimpleFoam} directory by typing:

\begin{quote}
\begin{verbatim}
$ cd $FOAM_RUN/elasticBodyDynamics/applications/pimpleFoam
\end{verbatim}
\end{quote}

\noindent Compile the files in the \textcolor{blue}{elasticBodyDynamics/applications/pimpleFoam} directory by typing the following in the terminal prompt:

\begin{quote}
\begin{verbatim}
$ wmake
\end{verbatim}
\end{quote}

\noindent After the compilation is successfully complete, an application named as \textcolor{blue}{newPimpleFoam} will be generated in the directory \textcolor{blue}{\$FOAM\_USER\_APPBIN}.



